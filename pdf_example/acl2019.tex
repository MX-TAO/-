%
% File acl2019.tex
%
%% Based on the style files for ACL 2018, NAACL 2018/19, which were
%% Based on the style files for ACL-2015, with some improvements
%%  taken from the NAACL-2016 style
%% Based on the style files for ACL-2014, which were, in turn,
%% based on ACL-2013, ACL-2012, ACL-2011, ACL-2010, ACL-IJCNLP-2009,
%% EACL-2009, IJCNLP-2008...
%% Based on the style files for EACL 2006 by 
%%e.agirre@ehu.es or Sergi.Balari@uab.es
%% and that of ACL 08 by Joakim Nivre and Noah Smith

\documentclass[11pt,a4paper]{article}
\usepackage[hyperref]{acl2019}
\usepackage{times}
\usepackage{latexsym}
\usepackage{amsfonts,amssymb}

\usepackage{url}

\aclfinalcopy % Uncomment this line for the final submission
%\def\aclpaperid{***} %  Enter the acl Paper ID here

%\setlength\titlebox{5cm}
% You can expand the titlebox if you need extra space
% to show all the authors. Please do not make the titlebox
% smaller than 5cm (the original size); we will check this
% in the camera-ready version and ask you to change it back.

\newcommand\BibTeX{B\textsc{ib}\TeX}

\title{A WSD System and a POS Tagger for a Rare Language}

\author{Mingxu TAO \\
  School of Electronics Engineering and Computer Science \\
  Peking University \\
  \texttt{kuroko@pku.edu.cn} }

\date{}

\begin{document}
\maketitle

\begin{abstract}
In the paper, we introduce a method to transfer words to vectors at first. Based on 
this word-to-vector method, we detail a translation model to judge whehter a sentence in 
English and a sentence in this rare language have the same meaning or not. Then, we can 
use the result to build a WSD and a POS system.
\end{abstract}


%% Start
\section{A Word to Vector Model}

In this section, we will introduce a model to encode words in vectors. 
Different from some \texttt{word2vec} models based on a large quantity of 
sentence-level train data, this model has no access to any sentence of 
this rare language. Therefore, we can not design a model straightly relied 
on corpus. Fortunately, a \texttt{WordNet-style} dictionary is provided. 
Since we can judge the similarity of any two recorded words via the dictionary, 
we can design two function $f$ and $s$, while $f$ encodes a word to a D-dim 
vector and $s$ evaluates the similarity of two word-vectors. 

\subsection{The Form of the Vectors}

For convince, it is necessary to limit the form of the word-vectors. Because 
of the continuity and compact of real numbers, we can use vectors within $[0,\,1]^{D}$ 
to represent infinite words theoretically. Therefore, the encoding function $f$ has form:
$$f: word\mapsto v\in [0,\,1]^{D}$$
while $D$ is a fixed integer.

In the meantime, $s$ should have this form:
$$s: \left \langle [0,\,1]^{D},\,[0,\,1]^{D}\right\rangle \mapsto \mathbb{R}$$

\subsection{Pre-train the Vectors}

If $D$ is equal to the number of words, we can simply use \texttt{one-hot} vectors 
to represent these words, for we can let words with the same meaning encode in the 
same vector, and different vectors otherwise. However, it is a waste of computer memory.
Usually, $D$ is a not too large number, and we can design a wise encode model to promise 
that the distance between the vectors of words evaluates the difference of them at most time
(a small amount of mistakes is inevitable). 

Now, imagine the words are some atoms in a box-shaped space. There is repulsion within 
some atoms, which means these words have different meanings, meanwhile, there is also 
attraction representing that some words have similar meanings. Although these atoms may
be at random location at first. With the effect of repulsion and attraction, they will 
finally stop at fixed locations.

After explaining the physical background, we will introduce some pre-set parameters: 
$N$ for the number of words in total, $r\in [0,\,1]$ for the ratio of base-point word to all, 
$L>0$ for the length to move per step.

First of all, we set all vectors at random loaction within $[0,\,1]^{D}$. 

Then, in every 
step, we choose $rN$ vectors as base-points randomly. Based on the effect of the forces 
of these $rN$ vectors, we will renew the locations of the other $(1-r)N$ vectors.

Just like algorithm \textbf{Simulated Annealing}, to imitate the real physical world 
can be efficent. Therefore, we choose the form of Newton's gravitation formula to design 
our renew mechanism. The force from $v_{1}$ to $v_{2}$: 
$$\overrightarrow{F}(v_{1},\,v_{2})=\frac{\overrightarrow{v_{1}}-\overrightarrow{v_{2}}}{\left \| \overrightarrow{v_{1}}-\overrightarrow{v_{2}} \right \| ^{2}}(-1)^{m}$$

if \texttt{WordNet} shows $w_{1}$ and $w_{2}$ have different meanings, then $m=1$, 
otherwise, $m=0$.

In step $t$, we choose series of base-points $\{v_{b_{1}},\,v_{b_{2}},\,\cdots,\, v_{b_{rN}}\}$ first, 
then renew the others (e.g. renew $v_{k}$):
$$v_{k}^{(t)}=v_{k}^{(t-1)}+L\cdot\sum_{i=1}^{rN}F(v_{b_{i}}^{(t-1)},\,v_{k}^{(t-1)})$$
$$v_{k}^{(t)}=max\{0,\,min\{1,\,v_{k}^{(t)}\}\}$$

We recommend that $L$ should be a not too large number, such as $10^{-3}$ or $10^{-4}$.

The parameter $r$ can affect the speed of convergence, when $r$ approaches to $1$, it 
will need a long time to converge, and when $r$ approaches to $0$, it may not converge.
Therefore, an $r$ between $0.3$ and $0.5$ may be proper.

If $D$ is too small, the atoms will be bound in a crowded box, so that they can not move 
by even one micron. If $D$ is too large, it will need more time to converge, and it may be 
a waste of computer memory sources. Some experience shows that, for millions of words, 
$D=300$ works well.

\subsection{Similarity Function}

We define $s$ with cosine and sigmoid:
$$s(v_{1},\,v_{2})=\sigma\left(\frac{v_{1}\cdot v_{2}}{\left\|v_{1}\right\|\cdot\left\|v_{2}\right\|}+s_{0}\right)$$

while $s_{0}$ is a automatic adjustable parameter, which should keep most $s(v_{1},\,v_{2})$ reflect the true result (positive for similar, negative for different).

\subsection{Unknown Words}

Since the \texttt{WordNet} dictionary might not contain all words of this rare language, 
it is necessary to have a certain method to propress the unknown words.

A simple method is to assume unknown words have blended features of all known words,
so we set: $v_{unknown}=\frac{1}{N}\sum_{i=1}^{N}v_{i}$. Sometimes, $v_{unknown}$ might not
be an all-zero vector.

\section{WSD System ans POS Tagging}

In the previous section, we introduce a word-to-vector method, and in this section, 
we will use these vectors to develop a semi-supervised learning models for WSD and POS.

Before detailing the models, we should stress the help of \texttt{GloVe}\textbf{(Pennington et al., 2014[1])}. %\cite{ACM:83}
\texttt{GloVe} is a reliable model to work on English word embedding, and we can easily 
have an access to some pre-trained English word embedding result. 

With the model in Section 1 and \texttt{GloVe}, we can build a translation model, and build
WSD and POS based on the translation model.

\subsection{Fake Sentence Generator}

Because we have no access to any linguistic material to that language, we have no idea about 
its syntax structure and the grammer. Thus, it is the only thing we can do to gain train data 
that we should build a sentence gennrator.

Now, we have lots of sentences in English. Via \texttt{WordNet}, we can translate English sentences
to new-language sentences word-by-word. These fake sentences may have a wrong syntax structure, 
so we can not use time-series model like LSTM. In this case, CNN pooling can be useful.

Assume an English sentence is $\{e_{i_{1}},\,e_{i_{2}},\,\cdots,\,e_{i_{k}}\}$, we can 
generate two kinds of fake sentences. True sentence translated word-by-word, 
$\{d_{i_{1}},\,d_{i_{2}},\,\cdots,\,d_{i_{k}}\}$; and the wrong sentence containing randomly 
selected $k$ words, $\{d_{j_{1}},\,d_{j_{2}},\,\cdots,\,d_{j_{k}}\}$.

\subsection{WSD System}

We will detail a three-stage model for WSD, this model is used to judge whether an English sentence 
and a new-language sentence are have the same meaning or not. 

We suppose $\{e_{i_{1}},\,e_{i_{2}},\,\cdots,\,e_{i_{k}}\}$ and $\{d_{x_{1}},\,d_{x_{2}},\,\cdots,\,d_{x_{k}}\}$ 
($x$=$i$/$j$) are given to judge, then the three stages should be as below ($e$ and $d$ are embedded vectors):

Pooling stage:
$$\widetilde{e}_{i_{t}}=\frac{1}{2h+1}\sum_{y=t-h}^{t+h}e_{i_{y}}$$

while $h$ is pooling step length, and do the same process to $\{d_{x}\}$.

Convolution stage:
$$\widetilde{h}_{e,t}=W_{e}\cdot\widetilde{e}_{i_{t}}$$
$$\widetilde{h}_{d,t}=W_{d}\cdot\widetilde{d}_{x_{t}}$$

while $W_{e}\in \mathbb{R}^{d\times D_{e}}$ and $W_{d}\in \mathbb{R}^{d\times D_{d}}$ are both trainable matrices.

Decode stage:
$$\widetilde{H}_{e}=\left[\widetilde{h}_{e,1},\,\widetilde{h}_{e,2},\,\cdots,\,\widetilde{h}_{e,k}\right]$$
$$\widetilde{H}_{d}=\left[\widetilde{h}_{d,1},\,\widetilde{h}_{d,2},\,\cdots,\,\widetilde{h}_{d,k}\right]$$
$$score=\frac{1}{d}tr\left( \sigma\left(\widetilde{H}_{e}\cdot\widetilde{H}_{d}^{T}\right) \right)$$

while $tr$ for trace, $\sigma$ for sigmoid. We ecpect $score=1$ when $x=i$, and $score=0$ for $x=j$. 
To train the model, we can use cross entropy to optimize $score$.

After training, we will use the model to build WSD system. When we receive 
a sentence $\{d_{i_{1}},\,d_{i_{2}},\,\cdots,\,d_{i_{k}}\}$, we can translate 
it to several English sentence word by word via \texttt{WordNet}, because of 
polysemants. Then, we can input these several pairs into the model and get 
some scores. Finally, we choose the English sentence with the highest score 
as the WSD result for that given rare-language sentence.

\subsection{POS tagging}

Compared with WSD introduced above, we also design a three-stage model.
In a brief, we do some revisions on the last model. We assume there is $P$ kinds of 
tags in English, so the model is changed as below:

Pooling: same as WSD

Convolution: fix $d=P$

Decode:
$$\widetilde{H}_{e}=\left[\widetilde{h}_{e,1},\,\widetilde{h}_{e,2},\,\cdots,\,\widetilde{h}_{e,k}\right]$$
$$\widetilde{H}_{d}=\left[\widetilde{h}_{d,1},\,\widetilde{h}_{d,2},\,\cdots,\,\widetilde{h}_{d,k}\right]$$
$$tag=\sigma \left( SVD\left(W_{p}\cdot\widetilde{H}_{e}\cdot\widetilde{H}_{d}^{T}\right)\right)$$

while SVD for Singular Value Decomposition. In addition, $W_{p}\in \mathbb{R}^{P\times P}$ 
is a trainable matrix. Then transfer $tag$ to a one-hot vector, which 
represents a kind of tagging.

\section{Model Explanation}

Compared with the traditional methods based on statistics and Bayesian formula and Markov Chain, 
this model innovate ideas of \textbf{word embedding and convolution}.

The \textbf{word vectors}, to some extent, can perform better to measure the similarity between the 
meanings of two words, for a real number is more accurate than a 0-or-1 boolean.

As it is detailed in the WSD and POS model, \textbf{convolution} can synthesize more information 
in the context, especially for the unknown words. When we use Bayesian method, for the 
unknown words have never appeared in train data, the Bayesian method will do mistakes all 
the time. However, our model can work in this circumstance.

\section{Which one is easier to build?}

Comparing WSD model in Section 2.2 and POS model in Section 2.3, it is obvious that WSD model 
is easier to trained. There is only one layer with $d\times(D_{e}+D_{d})$ parameters 
to train in WSD, while two layers with $p\times(D_{e}+D_{d})+p\times p$ parameters in POS.

Training WSD will need less data than POS, and the former will be easier and faster to converge.

Meanwhile, the information in \texttt{WordNet}(WSD basing on) is more reliable than 
syntax analysis(POS basing on).

\textbf{In total, it is easier to build WSD system for this rare language than POS.}

\section{References}

[1] Jeffrey Pennington, Richard Socher, Christopher D. Manning. \textbf{GloVe: Global Vectors for Word Representation}. Proceedings of the 2014 Conference on Empirical Methods in Natural Language Processing (EMNLP), pages 1532–1543.

\end{document}
